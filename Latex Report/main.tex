\documentclass[a4paper, 12pt]{article}
\usepackage{geometry}
\geometry{left=1in, right=1in, top=1in, bottom=1in}
\usepackage{graphicx}
\usepackage{amsmath}
\usepackage{setspace}
\setstretch{1.2}

\begin{document}

% Title Page
\begin{titlepage}
    \centering
    \vspace*{1in}
    \Huge
    \textbf{Analysis of Online Shoppers Intention Dataset}
    \vspace{1.5in}
    
    \LARGE
    \textbf{Author:} \\
    [Your Name] \\

    \vspace{0.5in}

    \textbf{Date:} \\
    [Submission Date]

    
    \vfill
\end{titlepage}

\tableofcontents
\newpage

\section{Introduction}

\subsection{Objective}
This report aims to analyze factors influencing online shoppers' purchasing intentions using the Online Shoppers Intention Dataset. The dataset includes quantitative, binary, and multiclass categorical variables describing user interactions within an online shopping platform. By applying data preprocessing and Principal Component Analysis, the goal is to uncover patterns that explain a user's likelihood to make a purchase.

\subsection{Scope}
This report explores the dataset, preprocesses the data, encodes it, normalizes for PCA, applies PCA, and interprets the results. Each step provides a systematic approach to handling a multivariate mixed dataset, with PCA results helping to identify key insights and reduce dimensionality.

\section{Dataset Description}

\subsection{Dataset Source}
\begin{itemize}
    \item \textbf{Source}: The Online Shoppers Intention Dataset comes from research on online shopping. This research looked at factors that affect whether people complete purchases.
    \item \textbf{Content}: The dataset contains information about individual online shopping sessions, including various behavioral and demographic metrics. It has 18 variables of different types, such as session duration, visitor type, and purchase outcome.
\end{itemize}

\subsection{Variable Overview}
The dataset contains a blend of quantitative, binary, and multiclass categorical variables:

\subsubsection{Quantitative Variables}
\begin{itemize}
    \item \textbf{Administrative}: Number of administrative pages visited.
    \item \textbf{Administrative\_Duration}: Time spent on administrative pages in seconds.
    \item \textbf{Informational}: Number of informational pages visited.
    \item \textbf{Informational\_Duration}: Time spent on informational pages in seconds.
    \item \textbf{ProductRelated}: Number of product-related pages visited.
    \item \textbf{ProductRelated\_Duration}: Time spent on product-related pages in seconds.
    \item \textbf{BounceRates}: Percentage of visitors leaving after one page.
    \item \textbf{ExitRates}: Percentage of sessions exiting from each page.
    \item \textbf{PageValues}: Average value attributed to a page.
    \item \textbf{SpecialDay}: Metric indicating proximity to significant holidays.
\end{itemize}

\subsubsection{Binary Variables}
\begin{itemize}
    \item \textbf{Weekend}: Indicates if the session occurred on a weekend (1 for Yes, 0 for No).
    \item \textbf{Revenue}: Indicates if the session resulted in a purchase (1 for Yes, 0 for No).
\end{itemize}

\subsubsection{Multiclass Categorical Variables}
\begin{itemize}
    \item \textbf{VisitorType}: Visitor category, such as \textit{Returning\_Visitor}, \textit{New\_Visitor}, or \textit{Other}.
    \item \textbf{Month}: Month of the visit (e.g., Jan, Feb).
    \item \textbf{OperatingSystems}: Visitor’s operating system.
    \item \textbf{Browser}: Browser used by the visitor.
    \item \textbf{Region}: Visitor’s geographical region.
    \item \textbf{TrafficType}: Type of traffic source leading to the visit.
\end{itemize}

\section{Data Preprocessing}

\subsection{Categorical Encoding}
\begin{itemize}
    \item Variables such as \textit{Month}, \textit{Region}, \textit{TrafficType}, \textit{VisitorType}, \textit{OperatingSystems}, and \textit{Browser} were converted into categorical types to retain the non-numeric nature of these features.
    \item Binary variables \textit{Weekend} and \textit{Revenue} were encoded as binary categories (1 for Yes, 0 for No).
\end{itemize}

\subsection{Data Summary}
Descriptive statistics and frequency distributions were calculated for the quantitative and categorical variables. This helped us understand the variable distributions and ranges before reducing the number of dimensions.

\subsection{Scaling Quantitative Variables}
Standardizing the numbers helped PCA work properly, since PCA depends on the differences between values.

\subsection{Summary Statistics for Quantitative Variables}

To understand the distribution of quantitative variables within the Online Shoppers Intention Dataset, summary statistics were calculated. These metrics include the mean, median, minimum, maximum, and standard deviation for each applicable variable.

\begin{itemize}
    \item \textbf{Administrative}: Mean = 2.3152, Median = 1, Std Dev = 3.3218, Min = 0, Max = 27
    \item \textbf{Administrative\_Duration}: Mean = 80.8186, Median = 7.5, Std Dev = 176.7791, Min = 0, Max = 3398.8
    \item \textbf{Informational}: Mean = 0.5036, Median = 0, Std Dev = 1.2702, Min = 0, Max = 24
    \item \textbf{Informational\_Duration}: Mean = 34.4724, Median = 0, Std Dev = 140.7493, Min = 0, Max = 2549.4
    \item \textbf{ProductRelated}: Mean = 31.7315, Median = 18, Std Dev = 44.4755, Min = 0, Max = 705
    \item \textbf{ProductRelated\_Duration}: Mean = 1194.7, Median = 598.9369, Std Dev = 1913.7, Min = 0, Max = 63974
    \item \textbf{BounceRates}: Mean = 0.0222, Median = 0.0031, Std Dev = 0.0485, Min = 0, Max = 0.2
    \item \textbf{ExitRates}: Mean = 0.0431, Median = 0.0252, Std Dev = 0.0486, Min = 0, Max = 0.2
    \item \textbf{PageValues}: Mean = 5.8893, Median = 0, Std Dev = 18.5684, Min = 0, Max = 361.7637
    \item \textbf{SpecialDay}: Mean = 0.0614, Median = 0, Std Dev = 0.1989, Min = 0, Max = 1
\end{itemize}
\end{document}